%%%%%%%%%%%
%% Home work template for Graduate School
%% Author : Thamme Gowda N.
%% Originally from  https://github.com/thammegowda/hw-tex-templ
%%%%%%%%%%%%%%

\documentclass[letterpaper,doc,notimes]{apa6}
%%\documentclass[tikz]{standalone}
\usepackage{tikz}
%Required by APA6 package
\usepackage[normalem]{ulem}
\usepackage[english]{babel}
\usepackage[utf8x]{inputenc}
\usepackage{amsmath}
\usepackage{amsfonts}
\usepackage{graphicx}
\usepackage{adjustbox}

%Oft-used, oft-abused
\usepackage{afterpage}
\usepackage{booktabs}
\usepackage{caption}
\usepackage{censor}
\usepackage{color}
\usepackage{csquotes}
\usepackage{enumitem}
\usepackage{float}
\usepackage{hyperref}
\usepackage{lmodern}
%\usepackage{media9}
\usepackage{multirow}
\usepackage{outlines}
\usepackage{pdfpages}
\usepackage{placeins}
\usepackage{soul}
\usepackage{tabularx}
\usepackage[colorinlistoftodos]{todonotes}
\usepackage{xcolor}
\usepackage{mathtools}

\usepackage{sectsty}
\sectionfont{\fontsize{16}{15}\selectfont}


\setenumerate[1]{label=\Roman*.}
\setenumerate[2]{label=\Alph*.}
\setenumerate[3]{label=\roman*.}
\setenumerate[4]{label=\alph*.}

% break page where ever is required
\allowdisplaybreaks

\title{ \textbf{ USC CSCI 567 HOMEWORK 2 SOLUTIONS} }
\shorttitle{USC CSCI567 FALL16 HW2}
\author{\textsc{ThammeGowda Narayanaswamy}}
\affiliation{ tnarayan@usc.edu \\ ID : 2074-6694-39 \\ Department of Computer Science \\ Viterbi School of Engineering \\ University of Southern California \\ Los Angeles, CA 
	}



%\note{September 13, 2016}
\note{\today}
\note{ Collaborators : \\ Ishan Alok}

\authornote{Produced for Fall 2016 session of CSCI 567, ``Machine Learning'', taught by Dr. Yan Liu at the University of Southern California}


\begin{document}

\maketitle
\newpage

\section{1. LOGISTIC REGRESSION}
\subsection{a. Negative Log Likelihood of Binary Logistic Regression}
The sigmoid-logistic function, defined by $\sigma(x) = \frac{1}{1 + e^{-x}}$ \\
Given that the dataset has $n$ examples, namely $\{(x_1,y_1), (x_2,y_2),... (x_n,y_n)\}$ \\
Where the class labels, $y_i = \{0, 1\}$  and $x_i$ $ \in \mathbb{R}^D$ \\
Let $w$ $ \in \mathbb{R}^D$ parameter which has to be learned by logistic regressor for optimal classification.\\
The probability estimation is defined by $P(Y=y_i|X=x_i) = \sigma(w^Tx_i) $. \\
The negative log likelihood is defined by:
\begin{align*}
		L(w) = & -log\bigg(\prod_{i=1}^{n} P(Y=y_i|X=x_i) \bigg) \\
		= & -\sum_{i=1}^{n}\bigg( log\big( P(Y=y_i|X=x_i) \big)\bigg) \\
		= & -\sum_{i=1}^{n}\bigg( log\big( P(Y=1|X=x_i)P(Y=0|X=x_i) \big)\bigg) \\
		= & -\sum_{i=1}^{n}\bigg( log\big[ \sigma(w^Tx_i)^{y_i} \times (1 - \sigma( w^Tx_i))^{1 - y_i} \big]\bigg) \\
		= & -\sum_{i=1}^{n}\bigg( y_i log\big[ \sigma(w^Tx_i) \big] + (1 - y_i ) log\big[ 1 - \sigma(w^Tx_i) \big]\bigg) \\
	%%	& w^Tx \text{ can also be expressed as sum of products, } \sum_{j}^{D} w_jx_j\\
\end{align*}

\section{1.b Find the update rule using gradient descent}
From the previous section, we have 
\begin{align*}
	E(w) = & \mathcal{L}(w) = -\sum_{i=1}^{n}\bigg( y_i log\big[ \sigma(w^Tx_i) \big] + (1 - y_i ) log\big[ 1 - \sigma(w^Tx_i) \big]\bigg) \\
	& \text{we know that $w$ is a multivariate function of variables } \{w_0, w_1, w_2, ....w_d\} \\
	&  w^Tx \text{ can also be expressed as sum of products, } \sum_{j}^{d} w_jx_j\\
	= & -\sum_{i=1}^{n}\bigg( y_i log\big[ \sigma(\sum_{j}^{d} w_jx_j) \big] + (1 - y_i ) log\big[ 1 - \sigma(\sum_{j}^{d} w_jx_j) \big]\bigg) \\
	& \text{We find the gradient of $E(w)$ w.r.t variable $w_j$ for $j = 0, 1, ...d $} \\ 
 \dfrac{\partial E(w)}{\partial w_j} = & -\sum_{i=1}^{n} \bigg(
			  y_i \dfrac{\partial}{\partial w_j} log\big[ \sigma(\sum_{j}^{d} w_jx_j) \big] 
			  + (1 - y_i ) \dfrac{\partial}{\partial w_j} log\big[ 1 - \sigma(\sum_{j}^{d} w_jx_j) 
		\big]\bigg) \\ 
		\\
\end{align*}
Let us evaluate the derivatives individually
\begin{align*}
\dfrac{\partial}{\partial w_j} log\big[ \sigma(\sum_{j}^{d} w_jx_j) \big]  = &
		 \dfrac{\partial}{\partial w_j} log\big[  \frac{1}{1 + e^{-\sum_{j}^{d} w_jx_j} } \big]\\
	= &  (1 + e^{-\sum_{j}^{d} w_jx_j}) \times \frac{-1}{(1 + e^{-\sum_{j}^{d} w_jx_j})^2} \times e^{-\sum_{j}^{d} w_jx_j} \times (-x_j) \\
	= &  \frac{1}{1 + e^{-\sum_{j}^{d} w_jx_j}} \times e^{-\sum_{j}^{d} w_jx_j} \times (x_j) \\
	= & x_j \times [1 - \sigma(\sum_{j}^{d} w_jx_j)] \\
	= & x_j \times [1 - \sigma(w^Tx)] \\
	\\
\dfrac{\partial}{\partial w_j} log\big[ 1 - \sigma(\sum_{j}^{d} w_jx_j) = &
	\dfrac{\partial}{\partial w_j} log\big[ 1 - \frac{1}{1 + e^{-\sum_{j}^{d} w_jx_j} } \big] \\
	= &  \frac{1}{1 - \frac{1}{1 + e^{-\sum_{j}^{d} w_jx_j}}} \times \frac{-1 \times -1 }{(1 + e^{-\sum_{j}^{d} w_jx_j})^2} \times e^{-\sum_{j}^{d} w_jx_j} \times (-x_j) \\
	= & -x_j \times \sigma(w^Tx) \\ 
\end{align*}

Putting the above two values in the original equation, we have:
\begin{align*}
\dfrac{\partial E(w)}{\partial w_j} = & -\sum_{i=1}^{n} \bigg(
	 y_i \dfrac{\partial}{\partial w_j} log\big[ \sigma(\sum_{j}^{d} w_jx_j) \big] 
	 + (1 - y_i ) \dfrac{\partial}{\partial w_j} log\big[ 1 - \sigma(\sum_{j}^{d} w_jx_j) 
	 \big]\bigg) \\
	 = & -\sum_{i=1}^{n} \bigg(	x_j y_i (1 - \sigma(w^Tx)) - (1 - y_i ) x_j \sigma(w^Tx) \bigg) \\
\end{align*}

The update rule using gradient descent is given by,

\begin{equation*}
	w_j = w_j - \alpha \dfrac{\partial E(w)}{\partial w_j}
\end{equation*}
{Where $\alpha$ is the learning or step rate.

Provided proper learning rate $\alpha$, Gradient descent optimizer converges to global minimum for all convex functions. So, to test the convexity of the negative log likelihood function, let us compute second order derivatives.

\begin{align*}
\dfrac{\partial^2 E(w)}{\partial w_j^2} = & \dfrac{\partial}{\partial w_j} \dfrac{\partial E(w)}{\partial w_j} \\
	= & -\sum_{i=1}^{n} \dfrac{\partial}{\partial w_j} \bigg(	x_j y_i (1 - \sigma(w^Tx)) - (1 - y_i ) x_j \sigma(w^Tx) \bigg) \\
	= & -\sum_{i=1}^{n} \bigg( x_j y_i (- \dfrac{\partial}{\partial w_j} \sigma(w^Tx)) - (1 - y_i ) x_j \dfrac{\partial}{\partial w_j} \sigma(w^Tx) \bigg) \\
	= & -\sum_{i=1}^{n} \bigg( x_j y_i (-x_j \sigma(w^Tx) (1 - \sigma(w^Tx))) - (1 - y_i ) x_j x_j  \sigma(w^Tx) (1 - \sigma(w^Tx)) \bigg) \\
	= & -\sum_{i=1}^{n} \bigg( [ -y_i - (1 - y_i ) ] x_j^2 \sigma(w^Tx) (1 - \sigma(w^Tx)) \bigg)  \\
	= & \sum_{i=1}^{n} \bigg([ y_i + 1 - y_i ) ] x_j^2 \sigma(w^Tx) (1 - \sigma(w^Tx)) \bigg)  \\
	= & \sum_{i=1}^{n} \bigg( x_j^2 \sigma(w^Tx) (1 - \sigma(w^Tx)) \bigg) \\
\end{align*}

We know that, $\sigma(w^Tx)$ is the $(P(Y=1|X=x))$ which is  >= 0, also, $(1 - \sigma(w^Tx))$ is the $(P(Y=0|X=x))$ which is also >= 0. \\
Thus $\dfrac{\partial^2 E(w)}{\partial w_j^2} >= 0 $. \\
Therefore, negative log likelihood function is convex.  Hence Gradient descent finds the global minimum.

\subsection{c. Multi class logistic regression model's Negative log likelihood}

Given that we have `K` classes, namely k=\{1,2,3,4,...K\}.

We know that the posterior probability of each class sums up to 1.
\begin{align*}
	\sum_{k=1}^{K} P(Y = k | X = x_i) = 1
\end{align*}

To reduce the number of required parameters, we can use the above result to precisely compute the probability of a class if we know rest all other classes.
\begin{align*}
	P(Y = K | X = x_i) = 1 - \sum_{k=1}^{K-1} P(Y = k | X = x_i)
\end{align*}

Let us define an Indicator function
 $I(y_i = k)$ which hs value $1$ when $y_i = k$, and has $0$ if $y_i \ne k$ 

For a dataset of $N$ examples, the likelihood of this multi class classifier (using fewer parameters ) is defined by
\begin{align*}
	l(w) = \prod_{n=1}^{N} \prod_{k=1}^{K-1} P(Y = y_n | X = x_n ) ^ {I(y_n=k)} (1 - \sum_{k=1}^{K-1} P(Y = y_n | X = x_n ) ^ {I(y_n=k)})
\end{align*}
We can simplify this equation without adding any additional parameters but setting $W_K=0$ as given in the problem description.

\begin{align*}
	\ell (w) = & \prod_{n=1}^{N} \prod_{k=1}^{K} P(Y = y_n | X = x_n )^{I(y_n = k)} & &
\end{align*}
The negative log likelihood is given by
\begin{align*}
  \mathcal{L}(w) = & - log \bigg [ \prod_{n=1}^{N} \prod_{k=1}^{K} P(Y = y_n | X = x_n )^{I(y_n = k)} \bigg]  & & \\
	   = & - \sum_{n=1}^{N} \sum_{k=1}^{K} log P(Y = y_n | X = x_n ) ^{I(y_n = k)}  & & \\
	   = & - \sum_{n=1}^{N} \sum_{k=1}^{K} log \bigg[ \frac{e^{w_k^Tx_n}}{1 + \sum_{t=1}^{K} e^{w_t^Tx_n}}  \bigg]^{I(y_n = k)}  & & \\
	   = & - \sum_{n=1}^{N} \sum_{k=1}^{K} I(y_n = k) \bigg[ w_k^Tx_n - log [1 +  \sum_{t=1}^{K} e^{w_t^Tx_n} ] \bigg]  & & \\
\end{align*}

\subsection{d. Gradient descent update rule}
Let us consider the final equation of previous section.
The $\mathcal{L}$ is a multivariate function with variables $w_i$ for i = 1, 2,3, ... K . \\

\begin{align*}
	\mathcal{L}(w_1, w_2, ...w_K ) = & - \sum_{n=1}^{N} \sum_{k=1}^{K} I(y_n = k) \bigg[ w_k^Tx_n - log [1 +  \sum_{t=1}^{K} e^{w_t^Tx_n} ] \bigg] 
\end{align*}
The first order derivative of $\mathcal{L}$ w.r.t $w_i$ is:

\begin{align*}
\frac{\partial \mathcal{L}}{\partial w_i} = & - \frac{\partial }{\partial w_i } \sum_{n=1}^{N} \sum_{k=1}^{K} I(y_n = k) \bigg[ w_k^Tx_n - log [1 +  \sum_{t=1}^{K} e^{w_t^Tx_n} ] \bigg ] \\
= & - \sum_{n=1}^{N} \sum_{k=1}^{K} I(y_n = k) \bigg[ x_{ni}  - \frac{1}{1 + \sum_{t=1}^{K} e^{w_t^Tx_n}} \times \sum_{t=1}^{K} e^{w_t^Tx_n} \times x_{ni} \bigg] \\
& \text{$x_n$ is a vector of all atttributes of $n^{th}$ example} 
\\ & \text{where as $x_{ni}$ is a scalar from $i^{th}$ attribute of $n^{th}$ example} \\
\\
= & - \sum_{n=1}^{N} \sum_{k=1}^{K} x_{ni} I(y_n = k) \bigg[ 1  - \frac{\sum_{t=1}^{K} e^{w_t^Tx_n}}{1 + \sum_{t=1}^{K} e^{w_t^Tx_n}}  \bigg] \\
\frac{\partial \mathcal{L}}{\partial w_i} = & - \sum_{n=1}^{N} \sum_{k=1}^{K} x_{ni} I(y_n = k) \bigg[ \frac{1}{1 + \sum_{t=1}^{K} e^{w_t^Tx_n}}  \bigg] \\
\end{align*}
The update rule for gradient descent optimizer with step rate $\alpha$ is: 
\begin{align*}
	w_i = & w_i - \alpha \frac{\partial \mathcal{L}}{\partial w_i} \\
	& \text{for all } i = \{1, 2, 3,...K\}
\end{align*}



%%%%%%%%%%%%%%%%%%%%%%%%%%%%%%%%%%%%%%%%%%%%%%%%%%%%%%%%%%%%%%%%%%%%%%%%
\section{2. LINEAR/GAUSSIAN DISCRIMINANT ANALYSIS}

\subsection{a. Log likelihood and MLE estimates}
Given:

The dataset $\mathcal{D}$ has $N$ examples, $\{(x_1,y_1), (x_2, y_2), ...(x_N, y_N)\}$

There are two classes $y_n = \{1, 2\} $

$P(y_n = 1) = p_1$ and $P(y_n = 2) = p_2$

$P(x_n | y_n=1) = \frac{1}{\sqrt{2\pi } \cdot \sigma_1} \exp(-\frac{(x_n-\mu_1)^2}{2\sigma_1^2})$

$P(x_n | y_n=2) = \frac{1}{\sqrt{2\pi } \cdot \sigma_2} \exp(-\frac{(x_n-\mu_2)^2}{2\sigma_2^2})$ \\

Let us define Indicator functions $I(y_n=1)$ and $I(y_n=2)$


\paragraph{Log Likelihood}
The log likelihood is defined by
\begin{align*}
\mathcal{L} = & ln \bigg[ \prod_{n=1}^{N} P(x_n, y_n) \bigg] \\
	= &  \ln \bigg[ \prod_{n=1}^{N} P(x_n, y_1)^{I(y_n=1)} \times P(x_n, y_1)^{I(y_n=2)} \bigg] \\
	= & \ln \bigg[ \prod_{n=1}^{N} P(x_n, y_1)^{I(y_n=1)} \times P(x_n, y_1)^{I(y_n=2)} \bigg] \\
	= &  \sum_{n=1}^{N}\bigg[ ln P(x_n, y_1)^{I(y_n=1)} + ln P(x_n, y_1)^{I(y_n=2)} \bigg] \\
	= &  \sum_{n=1}^{N}\bigg[ I(y_n=1) \ln \big[ p_1 \times \frac{1}{\sqrt{2\pi } \cdot \sigma_1} \exp(-\frac{(x_n-\mu_1)^2}{2\sigma_1^2}) \big] 
			 + I(y_n=2) \ln \big[ p_2 \times \frac{1}{\sqrt{2\pi } \cdot \sigma_2} \exp(-\frac{(x_n-\mu_2)^2}{2\sigma_2^2}) \big] \bigg]  \\
	= & \sum_{n=1}^{N}\bigg[
		I(y_n=1)\big[ \ln(p_1) -\frac{1}{2}\ln(2\pi) - \ln(\sigma_1) - \frac{1}{2\sigma_1^2} (x_n-\mu_1)^2 \big] \\
		 & \indent  + 
		I(y_n=2)\big[ \ln(p_2) -\frac{1}{2}\ln(2\pi) - \ln(\sigma_2) - \frac{1}{2\sigma_2^2} (x_n-\mu_2)^2 \big] \bigg] \\
\end{align*}

The above likelihood expression is a multivariate function with parameters $(p1, p2, \mu_1, \mu_2, \sigma_1, \sigma_2 )$.
From a fundamental law of probability, we know that
\begin{align*}
	p_1 + p_2 = & 1\\
	 p_2 = & 1 - p_1 \\
	 p_1 = & 1 - p_2
\end{align*}
Thus,
\begin{multline}
	\mathcal{L} (p1, \mu_1, \mu_2, \sigma_1, \sigma_2 )	=  \sum_{n=1}^{N}\bigg[
	I(y_n=1)\big[ \ln(p_1) -\frac{1}{2}\ln(2\pi) - \ln(\sigma_1) - \frac{1}{2\sigma_1^2} (x_n-\mu_1)^2 \big] \\
		\indent \indent \indent + 
	I(y_n=2)\big[ \ln(1 - p_1) -\frac{1}{2}\ln(2\pi) - \ln(\sigma_2) - \frac{1}{2\sigma_2^2} (x_n-\mu_2)^2 \big] \bigg] \\
\end{multline}
Let us find an estimate of these parameters using maximum likelihood estimator method.\\\
To find an estimate of $p_1$, we solve $\frac{\partial \mathcal{L} }{\partial p_1} = 0$
\begin{align*}
\frac{\partial \mathcal{L} }{\partial p_1} =& 0 \\
\sum_{n=1}^{N}\big[I(y_n=1)\frac{1}{p_1}  + I(y_n=2)\frac{1}{1-p_1} \times(-1) \big]  =& 0 \\ 
\frac{1}{p_1} \sum_{n=1}^{N} I(y_n=1)  - \frac{1}{1-p_1} \sum_{n=1}^{N} I(y_n=2) =& 0 \\ 
(1-p_1) \sum_{n=1}^{N} I(y_n=1)  - p_1 \sum_{n=1}^{N} I(y_n=2) =& 0 \\ 
p_1 =&  \frac{\sum_{n=1}^{N} I(y_n=1) }{\sum_{n=1}^{N} I(y_n=1) + \sum_{n=1}^{N} I(y_n=2)}\\ 
\implies p_1^* =&  \frac{\sum_{n=1}^{N} I(y_n=1) }{N} \\ 
\text{Similarly,} \\
1 - p_1 =& 1 - \frac{\sum_{n=1}^{N} I(y_n=1) }{N} = \frac{N - \sum_{n=1}^{N} I(y_n=1) }{N} \\ 
\implies p_2^* =&  \frac{\sum_{n=1}^{N} I(y_n=2) }{N} \\ 
\end{align*}

To find an estimate of $\mu_1$ when $\mathcal{L}$ is maximum, let us solve for $\frac{\partial \mathcal{L} }{\partial \mu_1} = 0$
\begin{align*}
\frac{\partial \mathcal{L} }{\partial \mu_1} =& 0 \\
	\sum_{n=1}^{N} I(y_n=1) \times \frac{-1}{2\sigma_1} \times 2 \times (x_n - \mu_1) \times -1 =& 0 \\
	\sum_{n=1}^{N} I(y_n=1) \times (x_n - \mu_1) =& 0 \\
		\sum_{n=1}^{N} I(y_n=1) x_n =&  \sum_{n=1}^{N} I(y_n=1) \mu_1 \\
				\sum_{n=1}^{N} I(y_n=1) x_n =& \mu_1 \sum_{n=1}^{N} I(y_n=1) \\
				\frac{ \sum_{n=1}^{N} I(y_n=1) x_n } {\sum_{n=1}^{N} I(y_n=1)}=& \mu_1  \\
				\implies \mu_1^* = & \frac{ \sum_{n=1}^{N} I(y_n=1) x_n } {\sum_{n=1}^{N} I(y_n=1)}  \\
		\text{We already know that, }  p_1 \times N = \sum_{n=1}^{N} I(y_n=1) & \\
	 \implies \mu_1^* = & \frac{ \sum_{n=1}^{N} I(y_n=1) x_n } {p_1 N} \\
	 \\
 \text{$\mu_2$ when $\mathcal{L}$ is maximum can be estimated by solving} & \\ \frac{\partial \mathcal{L} }{\partial \mu_2} = &0 \\
	 \implies \mu_2^* = & \frac{ \sum_{n=1}^{N} I(y_n=2) x_n } {\sum_{n=1}^{N} I(y_n=2)}  \\
	 	  = & \frac{ \sum_{n=1}^{N} I(y_n=2) x_n } {p_2 N} \\
\end{align*}

To find an estimate of $\sigma_1$ when $\mathcal{L}$ is maximum, let us solve
\begin{align*}
	\frac{\partial \mathcal{L} }{\partial \sigma_1} = &0  \\
	\sum_{n=1}^{N} I(y_n=1) \big[ -\frac{1}{\sigma_1} -\frac{(x_n - \mu_1)^2}{2} \times \frac{-2}{\sigma^3} \big] =& 0 \\
	\sum_{n=1}^{N} I(y_n=1) \big[ -\sigma_1^2 + (x_n - \mu_1)^2 \big] =& 0 \\
	\sigma_1^2 \sum_{n=1}^{N} I(y_n=1) =& \sum_{n=1}^{N} I(y_n=1) (x_n - \mu_1)^2 \\
    \sigma_1^2 =& \frac{ \sum_{n=1}^{N} I(y_n=1) (x_n - \mu_1)^2 }{\sum_{n=1}^{N} I(y_n=1)}\\
  \implies \sigma_1^* =& \sqrt{ \frac{ \sum_{n=1}^{N} I(y_n=1) (x_n - \mu_1)^2 }{\sum_{n=1}^{N} I(y_n=1)} } = \sqrt{ \sum_{n=1}^{N} \frac{ I(y_n=1) (x_n - \mu_1)^2 }{p_1 N} } \\
\end{align*}

Similarly, to find an estimate of $\sigma_2$ when $\mathcal{L}$ is maximum, let us solve

\begin{align*}
\frac{\partial \mathcal{L} }{\partial \sigma_2} = &0  \\
\implies \sigma_2^* =& \sqrt{ \frac{ \sum_{n=1}^{N} I(y_n=2) (x_n - \mu_2)^2 }{\sum_{n=1}^{N} I(y_n=2)} } = \sqrt{ \sum_{n=1}^{N} \frac{ I(y_n=2) (x_n - \mu_2)^2 }{p_2 N} } \\
\end{align*}


\subsection{b. Reduction to Logistic Sigmoid function form}
Given :\\
\indent  There are two classes, $y_n = \{c_1, c_2\}$ \\
\indent  $P(x | y= c_1) = \mathcal{N}(\mu_1, \sum)$  \\
\indent  $P(x | y=c_2) = \mathcal{N}(\mu_2, \sum) $ \\

\noindent We know \\
\indent for $D$ variables, \\
\indent \indent $\mathcal{N}(\mu, \sum) = (2\pi)^{-D/2} \sum^{-1/2} \exp[-\frac{1}{2} (x-\mu)^T \space \sum^{-1} \space (x-\mu)]$

\noindent Goal: \\
\indent  To express $P(y=c_1|x) $ in logistic sigmoid form

\noindent Solution:
\begin{align*}
\text{Using the Bayes rule} & \\
	P(y =& c_1|x) = \frac{P(x | y=c1) P(y=c1)}{P(x)} \\
\text{Using the law of otal probability} \\
	P(y =& c_1|x) = \frac{P(x | y=c1) P(y=c1)}{P(x | y=c1) P(y=c1) + P(x | y=c2) P(y=c2)} \\
	P(y =& c_1|x) = \frac{1}{1 + \frac{ P(x | y=c2) P(y=c2)}{P(x | y=c1) P(y=c1)} } \\
\text{This is of form} \\
	P(y =& c_1|x) = \frac{1}{1 + f(x)} \\
\text{where } f(x) = & \frac{P(y=c1)}{P(y=c1)} \times \frac{P(x|y=c2) }{P(x|y=c1)} \\
	 = & \frac{P(y=c1)}{P(y=c1)} \times \frac{\mathcal{N}(\mu_2, \sum) }{\mathcal{N}(\mu_1, \sum)}
\end{align*}
Now, substituting the values in $f(x)$
\begin{align*}
f(x) = & \frac{P(y=c1)}{P(y=c1)} \times \frac{(2\pi)^{-D/2} \sum^{-1/2} \exp[-\frac{1}{2} (x-\mu_2)^T \space \sum^{-1} \space (x-\mu_2)] }  {(2\pi)^{-D/2} \sum^{-1/2} \exp[-\frac{1}{2} (x-\mu_1)^T \space \sum^{-1} \space (x-\mu_1)]}\\ 
  = & \frac{P(y=c1)}{P(y=c1)} \times \exp \bigg[\frac{1}{2}  \big[(x-\mu_1)^T \space \sum^{-1} \space (x-\mu_1) - (x-\mu_2)^T \space \sum^{-1} \space (x-\mu_2)  \big] \bigg] \\
  \\ 
&  \text{For the sake of similification, we can use the form, } x = \exp(\ln(x))\\ 
\exp(\ln(f(x))) = &\exp(\ln\bigg( \frac{P(y=c1)}{P(y=c1)} \times \exp \bigg[\frac{1}{2}  \big[(x-\mu_1)^T \space \sum^{-1} \space (x-\mu_1) - (x-\mu_2)^T \space \sum^{-1} \space (x-\mu_2)  \big] \bigg] \bigg)) \\
\\
f(x) = &\exp(\ln(\frac{P(y=c1)}{P(y=c1)}) + \frac{1}{2} \big[(x-\mu_1)^T \space \sum^{-1} \space (x-\mu_1) - (x-\mu_2)^T \space \sum^{-1} \space (x-\mu_2)  \big] ) \\
f(x) = &\exp( - \big(- \ln(\frac{P(y=c1)}{P(y=c1)}) + \frac{1}{2} \big[ (x-\mu_2)^T \space \sum^{-1} \space (x-\mu_2) - (x-\mu_1)^T \space \sum^{-1} \space (x-\mu_1)\big] \big)) \\
\end{align*}
This equation is of  form $\exp(-(\theta_0 + {\theta}^Tx))$
Where, $\theta_0$ is a scalar computed by - $\ln(\frac{P(y=c1)}{P(y=c1)}) $ \\
 and $\theta^Tx$ is another scalar computed by $\frac{1}{2} \big[(x-\mu_2)^T \space \sum^{-1} \space (x-\mu_2) - (x-\mu_1)^T \space \sum^{-1} \space (x-\mu_1)  \big]$, but $\theta$ and $x$ are vectors of $D$ dimensions.
 
 Substituting the value of $f(x)$ in the original equation above, we get:
\begin{align*}
	P(y =& c_1|x) = \frac{1}{1 + f(x)} 
		= \frac{1}{1 + exp(-(\theta_0 + \theta^Tx))} 
 \end{align*}
 The above expression is in logistic sigmoid form.
 However, we can further simplify the above equation by packing $\theta_0$ scalar parameter inside $\theta$ vector by inserting $1$ into $x$ vector at the corresponding dimension. \\
 \begin{equation}
	\implies 	P(y = c_1|x) = \frac{1}{1 + exp(-\theta^Tx)}
 \end{equation}
 where $\theta$ and $x$ are vectors of $D+1$ dimensions

\end{document}