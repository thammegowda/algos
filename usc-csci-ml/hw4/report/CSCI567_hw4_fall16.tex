%%%%%%%%%%%
%% Home work template for Graduate School
%% Author : Thamme Gowda N.
%% Originally from  https://github.com/thammegowda/hw-tex-templ
%%%%%%%%%%%%%%

\documentclass[a4paper,doc,notimes]{article}
\usepackage[a4paper,margin=1in,footskip=0.25in]{geometry}
%%\documentclass[tikz]{standalone}
\usepackage{tikz}
%Required by APA6 package
\usepackage[normalem]{ulem}
\usepackage[english]{babel}
\usepackage[utf8x]{inputenc}
\usepackage{amsmath}
\usepackage{amssymb}
\usepackage{amsfonts}
\usepackage{graphicx}
\usepackage{adjustbox}

%Oft-used, oft-abused
\usepackage{afterpage}
\usepackage{booktabs}
\usepackage{caption}
\usepackage{censor}
\usepackage{color}
\usepackage{csquotes}
\usepackage{enumitem}
\usepackage{float}
\usepackage{hyperref}
\usepackage{lmodern}
%\usepackage{media9}
\usepackage{multirow}
\usepackage{outlines}
\usepackage{pdfpages}
\usepackage{placeins}
\usepackage{soul}
\usepackage{tabularx}
\usepackage[colorinlistoftodos]{todonotes}
\usepackage{xcolor}
\usepackage{mathtools}
\graphicspath{ {images/}}

\usepackage{sectsty}
\sectionfont{\fontsize{14}{12}\selectfont}

\setenumerate[1]{label=\Roman*.}
\setenumerate[2]{label=\Alph*.}
\setenumerate[3]{label=\roman*.}
\setenumerate[4]{label=\alph*.}

% break page where ever is required
\allowdisplaybreaks
\renewcommand{\thesubsection}{\thesection.\alph{subsection}}
\usepackage{titling}
\setlength{\droptitle}{-9em}
\newcommand\numberthis{\addtocounter{equation}{1}\tag{\theequation}}

\title{\noindent  \textbf{ USC CSCI 567 HOMEWORK 4 SOLUTIONS} }
\author{\textbf{ThammeGowda Narayanaswamy} \\
EMail: tnarayan@usc.edu  USCID: 2074-6694-39}


\begin{document}
\maketitle

\section{ BOOSTING}
Given : \newline
Labels: $y \in \{+1, -1\}$  \newline
Input features, $x_i \in \mathbb{R}^d$,  for $i = 1, 2, ... n$ \\
Weak learners, $\mathcal{H} = \{h_j, j=1,2,... M \} $ \\
Loss, $L(y_i, \hat{y_i}) = (y_i - \hat{y_i})^2$


\subsection {Gradient Calculation}
\begin{equation}\label{eqn:gradient}
	g_i = \frac{\partial L(y_i, \hat{y_i})} {\partial \hat{y_i}} = \frac{\partial }{\partial \hat{y_i}} (y_i - \hat{y_i})^2 = -2 (y_i - \hat{y_i}) 
\end{equation}
\subsection {Weak Learner Selection}
Given that the next learner is selected by, \\
\begin{align*}
	& h^* =  \arg \min_{h\in\mathcal{H}} \bigg(min_{\gamma\in \mathbb{R}} \sum_{i=1}^{n} (-g_i - \gamma h(x_i))^2\bigg) \numberthis \label{eqn:besth}\\
	& \text{Let } J =  \sum_{i=1}^{n} (-g_i - \gamma h(x_i))^2 \\
	& \text{The minimum of $J$ w.r.t to $\gamma$ is when its first order derivative is zero} \\
	&\frac{\partial }{\partial \gamma} \sum_{i=1}^{n} (-g_i - \gamma h(x_i))^2= 0 \\
	& 0 = \sum_{i=1}^{n} \frac{\partial }{\partial \gamma}(2 (y_i - \hat{y_i}) - \gamma h(x_i))^2  = \sum_{i=1}^{n}  2 (2 (y_i - \hat{y_i}) - \gamma h(x_i)) ( - h(x_i)) \\
	& 0 = - 2\sum_{i=1}^{n} h(x_i)(y_i - \hat{y_i})  + \gamma  \sum_{i=1}^{n} (h(x_i))^2 \\
	& \implies \gamma^* = \frac{2\sum_{i=1}^{n} h(x_i)(y_i - \hat{y_i}) }{\sum_{i=1}^{n} (h(x_i))^2 }
\end{align*}
By substituting the value of $\gamma*$ in \ref{eqn:besth}, we get
\begin{align*}
	h^* &=  \arg \min_{h\in\mathcal{H}} \bigg(\sum_{i=1}^{n} (-g_i - \gamma^* h(x_i))^2\bigg)  \\
	& = \arg \min_{h\in\mathcal{H}} \bigg(\sum_{i=1}^{n} \big[-g_i - \frac{2\sum_{j=1}^{n} h(x_j)(y_j - \hat{y_j}) }{\sum_{k=1}^{n} (h(x_k))^2 } \times h(x_i)\big ]^2\bigg)  \numberthis \label{eqn:besth2}\\
\end{align*}
Thus equation \refeq{eqn:besth2} can be derived independent of $\gamma$
\subsection {Step Size Selection}
Given that \\
\begin{align*}
	\alpha^* = & \arg \min_{\alpha \in \mathbb{R}} \sum_{i=1}^{n} \big[y_i - (\hat{y_i} + \alpha h^*(x_i)) \big]^2 \numberthis \\ \label{eqn:bestalpha1}
	& \text{Lets define } J = \sum_{i=1}^{n} \big[y_i - (\hat{y_i} + \alpha h^*(x_i)) \big]^2 \\
	& \text{$J$ is minimum w.r.t $\alpha$ when its first order derivative is zero.} \\
 0 = & \frac{\partial J}{\partial \alpha} = \sum_{i=1}^{n}  \frac{\partial }{\partial \alpha} \big[y_i - (\hat{y_i} + \alpha h^*(x_i)) \big]^2  = \sum_{i=1}^{n}  2 \big[y_i - (\hat{y_i} + \alpha h^*(x_i)) \big] (- h^*(x_i))\\
 0 = &   \sum_{i=1}^{n}  h^*(x_i)((\hat{y_i} - y_i) - \alpha  \sum_{i=1}^{n} (h^*(x_i) )^2 \\
 \implies \alpha^* =&  \frac{ \sum_{i=1}^{n}  h^*(x_i)((\hat{y_i} - y_i)}{\sum_{i=1}^{n} (h^*(x_i) )^2 } % \numberthis \label{eqn:bestalpha2}
\end{align*}
Thus, we can compute the step size analytically and perform the following update:
$$
	\hat{y_i} \leftarrow \hat{y_i} + \alpha^* h^*(x_i)
$$

\section{ NEURAL NETWORKS}
\subsection{}
Formulation: \\
Let us consider a neural network with $L$ layers. \\
The number of neurons in each layer are $\{n_1, n_2, ... n_{L_1} , 1\}$. Note that the input size is $n_1$ and output size is $1$.
As stated in the problem, the last layer has a single neuron with logistic activation. \\
All the neurons in the hidden layers $2, 3, ... L-1$ are having linear activation function \\
The following notation is used in the proof: \\
$a^{(l)}_k$ - activation of $k^{th}$  neuron in $l^{th}$ layer \\
$b^{(l)}_k$ - bias of $k^{th}$  neuron in $l^{th}$ layer  \\
$w^{(l)}_{ij}$ - weight for $i^{th}$  neuron in $l^{th}$ layer, has input from $j^{th}$ neuron of $(l-1)$ layer \\

In the first layer, $l=1$, activation is same as input, i.e., $a^{(1)}_j = x_j, j=1, 2, ... n_1$ \\ 

In the last layer, $l=L$, activation is output $y$, from a logistic function. \\
 i.e.,
 \begin{equation}\label{eqn:lastlayer}
	 y = a^{(L)}_1 = \sigma(\sum_{j=1}^{N_{L-1}} w^{(L)}_{1j} a^{(L-1)}_j + b^{(L)}_1 )
 \end{equation} 
 
$\forall $ Hidden layers, we have linear activations \\
\begin{equation}\label{eqn:hidden}
 a^{(l)}_{j} = \sum_{k=1}^{N_{l-1}} w^{(l)}_{jk} a^{(l-1)}_k + b^{(l)}_j , \text{  s.t.  }  2 \le l \le L-1; \forall j \in \{1, ... N_l\}
\end{equation}
Specifically, for $l = 2$, we have much nicer activation equation:
\begin{equation} \label{eqn:layer2}
	a^{(2)}_{j} =  \sum_{k=1}^{N_{1}} w^{(2)}_{jk} a^{(1)}_k + b^{(2)}_j =  \sum_{k=1}^{N_{1}} w^{(2)}_{jk} x_k + b^{(2)}_j
\end{equation}
By using the fact \ref{eqn:hidden} recursively in \ref{eqn:lastlayer} until we reach the base case \ref{eqn:layer2}
  \begin{align*}
 	 y = a^{(L)}_1 & = \sigma\bigg(\sum_{j=1}^{N_{L-1}} w^{(L)}_{1j} \big[ \sum_{k=1}^{N_{L-2}} w^{(L-1)}_{jk} a^{(L-2)}_k + b^{(L-1)}_j  \big] + b^{(L)}_1 \bigg) \\
 	 & =  \sigma\bigg(\sum_{j=1}^{N_{L-1}} w^{(L)}_{1j} \big[ \sum_{k=1}^{N_{L-2}} w^{(L-1)}_{jk} [ ... ( \sum_{i=1}^{N_{1}} w^{(2)}_{j'i} x_i + b^{(2)}_{j'} ) +  ... ] + b^{(L-1)}_j  \big] + b^{(L)}_1 \bigg) \\
	 	 & \text{NOTE: $j'$ is the connecting neuron's indice in 3rd layer}
 \end{align*}
 With algebraic manipulations we can expand the series and group the terms such that all terms with $x_i$ are at one place and similarly biases can also be grouped together.
 In the next step we note that all the terms common to $x_i$  can be reduced to $R_i \in \mathbb{R}$ and all the biases can be reduced to a real number $R_b \in \mathbb{R}$
 This forms the equation:
\begin{equation}
	 y = a^{(L)}_1  = \sigma(\sum_{i=1}^{N_1} R_i x_i + R_b)
\end{equation}
Thus the whole network is equivalent to a single logistic neuron.

\subsection{Back Propagation}
Given: A network with 1 hidden layer\\
The input layer, $x_i$ for $i=1,2,3$
Activation of the hidden layer, $z_k = \tanh(\sum_{i=1}^{3} w_{ki}x_i)$ \\
Activation of the output layer, $\hat{y_j} = \sum_{k=1}^{4} v_{jk}z_k$ for $j = 1, 2$ \\
The optimization function is a squared error function, given by
\begin{equation} \label{eqn:squaredloss}
	L(y_i, \hat{y_i}) = \frac{1}{2} \big( \sum_{j=1}^{2}(y_j - \hat{y_j})^2 \big) 
\end{equation}

\subsubsection{Updates for the last layer}
Lets find the gradient of equation \ref{eqn:squaredloss} w.r.t parameters of the last layer neurons.
\begin{equation}
 \frac{\partial L}{\partial v_{jk}} = \sum_{j=1}^{2}(y_j - \hat{y_j}) \times \frac{\partial \hat{y_j}}{\partial v_{jk}} = \sum_{j=1}^{2}(y_j - \hat{y_j}) \times \frac{\partial }{\partial v_{jk}} \sum_{k=1}^{4} v_{jk}z_k  = - \sum_{k=1}^{4} z_k \sum_{j=1}^{2}(y_j - \hat{y_j}) 
\end{equation}
Update to be made using step size $\alpha$: 
\begin{align*}
 \Delta v_{jk}  = & - \alpha \frac{\partial L}{\partial v_{jk}}  \\
 v_{jk} \leftarrow &  v_{jk}  + \Delta v_{jk} \numberthis
\end{align*}

\subsubsection{Updates for the hidden layer}
Lets find the gradient of equation \ref{eqn:squaredloss} w.r.t parameters of the hidden layer neurons.
\begin{equation}\label{eqn:wki_derive}
\frac{\partial L}{\partial w_{ki}} = \sum_{j=1}^{2}(y_j - \hat{y_j}) \times \frac{\partial \hat{y_j}}{\partial w_{ki}} = \sum_{j=1}^{2}(y_j - \hat{y_j}) \times \frac{\partial }{\partial w_{ki}} \sum_{k=1}^{4} v_{jk}z_k  = \sum_{j=1}^{2}(y_j - \hat{y_j}) \times \sum_{k=1}^{4} v_{jk}  \frac{\partial z_k}{\partial w_{ki}} 
\end{equation}
\begin{equation} \label{eqn:zk_deriv}
\frac{\partial z_k}{\partial w_{ki}} = \frac{\partial }{\partial w_{ki}}  \tanh(\sum_{i=1}^{3} w_{ki}x_i) = [ 1 - \tanh^2(\sum_{i=1}^{3} w_{ki}x_i)] \times  \frac{\partial }{\partial w_{ki}} \sum_{i=1}^{3} w_{ki}x_i = [ 1 - \tanh^2(\sum_{i=1}^{3} w_{ki}x_i)] \times   \sum_{i=1}^{3} x_i  
\end{equation}
Substituting \ref{eqn:zk_deriv} in \ref{eqn:wki_derive}, we get: 
\begin{equation}\label{eqn:wki_derive2}
\frac{\partial L}{\partial w_{ki}} = \sum_{j=1}^{2}(y_j - \hat{y_j}) \bigg[ \sum_{k=1}^{4} v_{jk} \big( 1 - \tanh^2(\sum_{i=1}^{3} w_{ki}x_i) \big)\times   \sum_{i=1}^{3} x_i  \bigg]
\end{equation}

Update to be made using step size $\alpha$ : 
\begin{align*}
\Delta w_{ki}  = & - \alpha \frac{\partial L}{\partial w_{ki}} \\
w_{ki} \leftarrow &  w_{ki}  + \Delta w_{ki} \numberthis
\end{align*}


\section{PROGRAMMING}
\stepcounter{subsection}
\stepcounter{subsection}
\stepcounter{subsection}

\subsection{Linear Activation}
Report the observations and explain the pattern of test set accuracies obtained. Also report the time taken to train these new set of architectures.

\subsection{Sigmoid Activation}
Report your test set accuracies and comment on the trend of accuracies obtained with changing model architectures. Also explain why this trend is different from that of linear activations. Report and compare the time taken to train these architectures with those for linear architectures.

\subsection{ReLu Activation}
Report your observations and explain the trend again. Also explain why this trend is different from that of linear activations. Report and compare the time taken to train these architectures with those for linear and sigmoid architectures.

\subsection{L2-Regularization}
Report your accuracies on the test set and explain the trend of observations. Report the best value of the regularization hyperparameter.

\subsection{Early Stopping and L2-regularization}
Again report your accuracies on the test set and explain the trend of observations. Report the best value of the regularization hyperparameter this time. Is it the same as with only L2-regularization? Did early stopping help?

\subsection{ SGD with weight decay}
Report your test set accuracies for the decay parameters and choose the best one based on your observations.

\subsection{Momentum}
Find the best value for the momentum coefficients, which gives the maximum test set accuracy.

\subsection{Combining the above}
Report your test set accuracy again. Is it better or worse than the accuracies you observed in the last few parts?

\subsection{Grid search with cross-validation}
Report the best parameter values, architecture and the best test set accuracy obtained.

\end{document}